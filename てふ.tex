\documentclass[a4paper]{jarticle}
\usepackage{amsmath,amsthm,amssymb}
\usepackage[dvipdfmx]{graphicx}
\usepackage{here}
\usepackage{ascmac}
\usepackage{url}


\title{挫折しないTeX 入門}
\author{山岸 敦\thanks{東京大学経済学部4年 E-mail: haru.su.jp at gmail.com}}
\date{2015/4/22}

\begin{document}

\maketitle

\section{\TeX ってなに\texttt{??} }
\TeX とは、主にアカデミックにおいて多用される文章作成ソフトです。松井ゼミは卒論必修ですが、{\bf 卒論は\TeX で書くことが期待されています}。あと、大学院行く人は将来間違いなく必要ですので、今のうちに最低限は習得しておくべきだと思います。

Wordなどと比べた比較優位として、1.数式が美しく出力できる2.すべての情報をコマンドで管理するので、統一的な書式にできる、といった点があります。数式出力の実例を出すと
\begin{multline}
V(y;p(i), i \in[0, N]) = \frac{a^2N}{2b} - a \int_{0}^{N}p(i)di + \\
 \frac{b + cN}{2}\int_{0}^{N}(p(i))^2di - \frac{c}{2}(\int_{0}^{N}p(i)di)^2 + y + \overline{q}_0
\end{multline}
とこんな感じ。めんどくさそうな式もしっかり綺麗に出力できます。

一方で、\TeX の使い方はプログラミング的で、敷居が高く見えるのが実情です\footnote{去年の僕はこんなような説明プリントを手に震えていました}。そこで、今回のプリントでは\TeX を使用し始めるにあたって挫折しない方法を自分の経験をもとに伝授します。これも\TeX で書いてますが、馴れると便利ですよ\footnote{文章の様式美が保てる、数式が出しやすい、というのが大きいです}。

\section{\TeX をインストールしよう\texttt{!!}}
インストールが最大の関門です。\TeX はフリーソフトなので無料でも頑張ればインストールできますが、一番手っ取り早いのは「\LaTeX $2\epsilon$ 美文書作成入門 第六版」by奥村、黒木にくっついてくるDVDを使うことです。このDVDは、\TeX を使いはじめるのにあたり必要なものを自動ですべてインストールしてくれるすぐれもので、トラブル遭遇確率が最も低いです。本自体も\TeX 利用時の辞書として使えるので持っておくことを推奨します。

どうしてもお金をつかいたくない、という人は上述の奥村晴彦先生のサイトに無料で\TeX 利用環境を整える方法について詳しく乗っているので参考に頑張ってみてください。

\section{\TeX のしくみについて:超簡略版}
\TeX 利用の流れは、
\begin{enumerate}
\item
コマンド(プログラム)をエディターに入力する。上述のDVDで導入した場合、TeXworksというソフトがそれ
\item
タイプセット(コンパイル)する。これはコマンドをパソコンに読み込ませる操作で、これが済むと書いてる文章が出力されます
\item
1と2を繰り返しながら、試行錯誤しつつ文章を完成させてください
\end{enumerate}


\section{\TeX を使い始めよう}
\TeX をインストールしたからといって、真面目に使い方をゼロから勉強するのは挫折まっしぐらです。ずぶの素人が何らかの文章を\TeX で作らなくてはいけないとき、大切なのは{\bf 使えそうなフリーのテンプレートを拾ってきてパクる}\footnote{"The only way you can get good, unless you're a genius, is to copy. That's the best thing. Just steal." by Ritchie Blackmore}ことです。正しいテンプレには、正しい\TeX の使い方が詰まっています。これをテキストやネットを駆使して使い方を調べながらいじることから始めましょう。そのうち、仕組みがわかってくるのでそうしたら必要に応じ本格的に勉強するほうがすっと頭に入ると思います。あまり質は良くないですが、僕のgithubページ\verb|https://github.com/haru110jp|にこのプリントのtexファイルを載せておきます。それをエディターにコピペしてコンパイルすると、これと同じものが出力されるはずなので、自分でいじって遊んでみてください。試行錯誤が上達のもとです。

\section{補論:BiBTeX}
参考文献をいちいち書くのがめんどくさい人は、BiBTeXというのを導入すると自動で参考文献リストを作ってくれます。研究者を考えてる人は導入すると便利かと思います。「武田史郎 BiBTeX」でググるといい解説が出てくるので、参考にどうぞ。

\end{document}